\documentclass[12pt]{article}
\usepackage{amsfonts}
\usepackage[hscale=.8,vscale=.8]{geometry}
\usepackage{hyperref}
\usepackage{amsmath}

\begin{document}

  \section{Model Setup}

    Consider an underlying asset following a geometric Brownian motion with time dependent but deterministic
    parameters, in the risk-neutral measure $\mathbb{Q}$ induced by the money market numeraire $\beta(t)=
    \exp\left(\int_0^tr(u)du\right)$,
    \begin{equation}
      \frac{dS(t)}{S(t)}=\left(r(t)-q(t)\right)dt+\sigma(t)dW^{\mathbb{Q}}(t),
    \end{equation}
    where $r$, $q$, and $\sigma$ are the risk free rate, the dividend yield, and the volatility, respectively.
    The forward function for the underlying can be explicitly calculated,
    \begin{equation}
      F(t)={\rm E}^{\mathbb{Q}}\left[S(t)\right]=S(0)\exp\left(\int_0^t\left(r(u)-q(u)\right)du\right).
    \end{equation}
    Using this deterministic forward function, the underlying dynamics can be reformulated in terms of
    $X(t)=\log\left(S(t)/F(t)\right)$, which leads to
    \begin{equation}
      dX(t) = -\frac{1}{2}\sigma(t)^2dt + \sigma(t)dW^{\mathbb{Q}}(t).
    \end{equation}

    For a European option with expiry $T$ and strike $K$, its discounted price is given by
    \begin{eqnarray}
      \label{Euro1}
      v_E(t,x)&\equiv&P(t)V_E(t,F(t)e^x;T,K,\omega)\nonumber\\
                         &=&\omega P(T)F(T)\bigg(e^x\Phi\big(\omega d_+\left(x-x^*,\Sigma(t,T)\right)\big)\nonumber\\
                         &&\ \ \ \ \ \ \ \ \ \ \ \ \ \ \ -e^{x^*}\Phi\big(\omega d_-\left(x-x^*,\Sigma(t,T)\right)\big)\bigg),
    \end{eqnarray}
    where $\omega=\pm 1$ for call/put options, $x^*=\log(K/F(T))$ is the moneyness of the option, $\Phi(x)$
    is the cumulative distribution function of the standard normal distribution, $P(t)=\beta(t)^{-1}$ is the discount
    factor at time $r$, and
    \begin{equation}
      d_{\pm}(x, v)=\frac{x}{v}\pm\frac{1}{2}v,
    \end{equation}
    with
    \begin{equation}
      \Sigma(t,T)^2=\int_t^T\sigma(u)^2du
    \end{equation}
    Also, the discounted European option price satisfies the following Black-Scholes equation,
    \begin{equation}
      \frac{\partial v_E(t,x)}{\partial t}-\frac{1}{2}\sigma(t)^2\frac{\partial v_E(t,x)}{\partial x}
         +\frac{1}{2}\sigma(t)^2\frac{\partial^2 v_E(t,x)}{\partial x^2}=0.
    \end{equation}



  \section{Early Exercise Premium Representation}

    Now, let us consider the valuation of American options. Due to the early exercise provision, it is preferable
    to exercise the option and receive the intric value of the option, when the underlying prices moves above or
    below the optimal exercise boundary for a call or put option. Otherwise, the American option is identical to
    its European counterpart. Denote the optimal exercise boundary as $B(t)$, and normalize it with respect to
    the forward, $X_T(t)=\log(B(t)/F(t))$, the discounted American option price can be written as
    \begin{equation}
      v_A(t,x(t))=P(t)V_A(t,F(t)e^{x(t)};T,K,\omega)
              =\begin{cases}
                 \omega P(t)\left(F(t)e^{x(t)}-K\right), & \mbox{if } \omega x(t) \geq \omega X_T(t),\\
                 v_E(t,x), & \mbox{otherwise}.
               \end{cases}
    \end{equation}
    The dynamics of the discounted American option price can be obtained by the application of Ito's lemma in the
    continuation and the exercise regions,
    \begin{eqnarray}
      dv_A(t,x(t))&=&\mathbf{1}_{\{\omega x(t) \geq \omega X_T(t)\}}\omega\bigg(P^{\prime}(t)\left(F(t)e^{x(t)}-K\right)dt
                                                                     +P(t)F^{\prime}(t)e^{x(t)}dt\nonumber\\
               &&\quad\quad\quad\quad\quad\quad\quad\quad +P(t)F(t)e^{x(t)}\left(-\frac{1}{2}\sigma(t)^2dt
                                                          + \sigma(t)dW^{\mathbb{Q}}(t)\right)\nonumber\\
               &&\quad\quad\quad\quad\quad\quad\quad\quad +P(t)F(t)e^{x(t)}\frac{1}{2}\sigma(t)^2dt\bigg)\nonumber\\
               && + \mathbf{1}_{\{\omega x(t)<\omega X_T(t)\}}\bigg[\left(\frac{\partial v_E(t,x)}{\partial t}
                        -\frac{1}{2}\sigma(t)^2\frac{\partial v_E(t,x)}{\partial x}
                    +\frac{1}{2}\sigma(t)^2\frac{\partial^2 v_E(t,x)}{\partial x^2}\right)dt\nonumber\\
               &&\quad\quad\quad\quad\quad\quad\quad\quad +\sigma(t)\frac{\partial v_E(t,x)}{\partial x}dW^{\mathbb{Q}}(t)\bigg].
    \end{eqnarray}
    Notice the Black-Scholes equation for the discounted European option price, we have
    \begin{eqnarray}
      dv_A(t,x(t))&=&\mathbf{1}_{\{\omega x(t)\geq\omega X_T(t)\}}\omega\bigg(\left(\left(P(t)F(t)\right)^{\prime}e^{x(t)}
                                                                          -P^{\prime}(t)K\right)dt
                                              +P(t)F(t)e^{x(t)} \sigma(t)dW^{\mathbb{Q}}(t)\bigg)\nonumber\\
               && + \mathbf{1}_{\{\omega x(t)<\omega X_T(t)\}}
                    \bigg[\sigma(t)\frac{\partial v_E(t,x)}{\partial x}dW^{\mathbb{Q}}(t)\bigg].
    \end{eqnarray}
    Integrate both sides, and take the risk neutral expectation, the integration with respect to the Brownian motion
    will drop out, we are left with
    \begin{eqnarray}
      &&{\rm E}_t^{\mathbb{Q}}\left[v_A(T,x(T))\right]-v_A(t,x(t)) \nonumber\\
      &&\quad=\omega\int_t^T\left(\left(P(u)F(u)\right)^{\prime}
               {\rm E}_t^{\mathbb{Q}}\left[\mathbf{1}_{\{\omega x(u)\geq\omega X_T(u)\}}e^{x(u)}\right]
               -P^{\prime}(u)K{\rm E}_t^{\mathbb{Q}}\left[\mathbf{1}_{\{\omega x(u)\geq\omega X_T(u)\}}\right]\right)du,
    \end{eqnarray}
    or equivalently,
    \begin{eqnarray}
      v_A(t,x(t))&=&{\rm E}_t^{\mathbb{Q}}\left[v_A(T,x(T))\right]-\omega\int_t^T\bigg(\left(P(u)F(u)\right)^{\prime}
               {\rm E}_t^{\mathbb{Q}}\left[\mathbf{1}_{\{\omega x(u)\geq\omega X_T(u)\}}e^{x(u)}\right]\nonumber\\
               &&\quad\quad\quad\quad\quad\quad\quad\quad\quad\quad\quad\quad\quad-P^{\prime}(u)K
               {\rm E}_t^{\mathbb{Q}}\left[\mathbf{1}_{\{\omega x(u)\geq\omega X_T(u)\}}\right]\bigg)du.
    \end{eqnarray}
    This formula is a generic result, and can be applied to any local volatility model.

    At expiry, the payoff of the American option coincides with the European option,
    \begin{equation}
      v_A(T,x(T)) = v_E(T,x(T))=P(T)\left[\omega\left(F(T)e^{x(T)}-K\right)\right]^+,
    \end{equation}
    the risk neutral expectation of which will actually reproduce the discounted European option price, Eq. (\ref{Euro1}).
    Also notice that, under our model setup,
    \begin{equation}
      {\rm E}_t^{\mathbb{Q}}\left[\mathbf{1}_{\{\omega x(u)\geq\omega X_T(u)\}}e^{x(u)}\right]
       = e^{x(t)}\Phi\left(\omega d_+\left(x(t)-X_T(u), \Sigma(t,u)\right)\right),
    \end{equation}
    and
    \begin{equation}
      {\rm E}_t^{\mathbb{Q}}\left[\mathbf{1}_{\{\omega x(u)\geq\omega X_T(u)\}}\right]
       = \Phi\left(\omega d_-\left(x(t)-X_T(u), \Sigma(t,u)\right)\right),
    \end{equation}
    then the discounted American option price can be represented as the sum of the discounted European
    option price and the early exercise premium,
    \begin{eqnarray}
      \label{EEP}
      v_A(t,x(t))&=&v_E(t,x(t))+\omega\int_t^T\bigg[P^{\prime}(u)K\Phi\left(\omega d_-\left(x(t)-X_T(u), \Sigma(t,u)\right)\right)\nonumber\\
                 &&\quad\quad\quad\quad\quad\quad\quad
                  -\left(P(u)F(u)\right)^{\prime}e^{x(t)}\Phi\left(\omega d_+\left(x(t)-X_T(u), \Sigma(t,u)\right)\right)\bigg]du.
    \end{eqnarray}

  \section{Optimal Exercise Boundary}

    The above representation of the American option price requires the knowledge of the optimal
    exercise boundary, which is unknown and must be solved in tandem with the computation of the American
    option price. From Eq. (\ref{EEP}), at the boundary, the American option price becomes its intrinsic
    value,
    \begin{eqnarray}
       \label{Boundary}
       && P(t)\left(F(t)e^{X_T(t)}-K\right)\nonumber\\
      =&& P(T)\bigg(F(T)e^{X_T(t)}\Phi\big(\omega d_+\left(X_T(t)-x^*,\Sigma(t,T)\right)\big)
                                                      -K\Phi\big(\omega d_-\left(X_T(t)-x^*,\Sigma(t,T)\right)\big)\bigg)\nonumber\\
       &&+\int_t^T\bigg[P^{\prime}(u)K\Phi\left(\omega d_-\left(X_T(t)-X_T(u), \Sigma(t,u)\right)\right)\nonumber\\
                 &&\quad\quad\quad\quad\quad
                  -\left(P(u)F(u)\right)^{\prime}e^{X_T(t)}\Phi\left(\omega d_+\left(X_T(t)-X_T(u), \Sigma(t,u)\right)\right)\bigg]du.
    \end{eqnarray}
    Using the relations
    \begin{equation}
      \int_t^T\left(P(u)F(u)\right)^{\prime}du=P(T)F(T)-P(t)F(t),
    \end{equation}
    and
    \begin{equation}
      \int_t^TP^{\prime}(u)du=P(T)-P(t),
    \end{equation}
    the optimal exercise boundary can be determined from the following equation,
    \begin{equation}
      e^{X_T(t)}=e^{x^*}\frac{N(X_T, \Sigma)}{D(X_T, \Sigma)},
    \end{equation}
    where the numerator $N$ and the denominator $D$ are functionals of the optimal exercise boundary,
    \begin{eqnarray}
      N(X_T, \Sigma)&=&\Phi\big(-\omega d_-\left(X_T(t)-x^*,\Sigma(t,T)\right)\big)\nonumber\\
      &&-\int_t^T\frac{P^{\prime}(u)}{P(T)}\Phi\big(-\omega d_-\left(X_T(t)-X_T(u),\Sigma(t,u)\right)\big)du,
    \end{eqnarray}
    and
    \begin{eqnarray}
      D(X_T, \Sigma)&=&\Phi\big(-\omega d_+\left(X_T(t)-x^*,\Sigma(t,T)\right)\big)\nonumber\\
      &&-\int_t^T\frac{\left(P(u)F(u)\right)^{\prime}}{P(T)F(T)}\Phi\big(-\omega d_+\left(X_T(t)-X_T(u),\Sigma(t,u)\right)\big)du.
    \end{eqnarray}

    From the above formulation, the optimal exercise boundary can be determined by iteration. At the start of
    each iteration, with the knowledge of the optimal exercise boundary $X_T^{(j)}(t)$ from the previous iteration,
    a new boundary $X_T^{(j+1)}(t)$ can be found. This process can be repeated until convergence is reached.





\begin{thebibliography}{99}
  \bibitem{ALO}
    Leif Andersen, Mark Lake and Dimitri Offengenden,
    \href{https://www.risk.net/journal-of-computational-finance/2464632/high-performance-american-option-pricing}
    {\it High-performance American option pricing},
    Journal of Computational Finance 20(1):39-87 (2016).
\end{thebibliography}


\end{document}
