\documentclass[12pt]{article}
\usepackage{amsfonts}
\usepackage[hscale=.8,vscale=.8]{geometry}
\usepackage{hyperref}
\usepackage{amsmath}

\begin{document}

  \section{Model Setup}

    Consider an underlying asset with the dynamics described by the following local volatility process,
    in the risk-neutral measure,
    \begin{equation}
      \frac{dS(t)}{S(t)}=-\frac{F^{\prime}(t)}{F(t)}dt+\sigma\left(t, S(t)\right)dW^{\mathbb Q}(t),
    \end{equation}
    where $F(t)$ and $\sigma(t, S)$ are the funding curve and the local volatility function, respectively.
    Therefore, the forward value of the underlying at time $T$ with spot value $S$ at time $t$ is given by
    $S/F(t,T)$, where $F(t,T)=F(T)/F(t)$.

    The discounted European option price is given by the risk neutral expectation of the terminal payoff,
    \begin{equation}
      \label{Euro0}
      v_E(t,S) \equiv P(t)V_E(t,S;T,K,\omega) = {\mathbb E}_t^{\mathbb Q}\left[P(T)\left(\omega[S(T)-K]\right)^+\right],
    \end{equation}
    where $\omega=\pm 1$ for call/put options and $P(t)$ is the discount curve. The discounted option
    price can also be solved from the associated Black-Scholes equation,
    \begin{equation}
      \frac{\partial v_E(t,S)}{\partial t}-\frac{F^{\prime}(t)}{F(t)}S\frac{\partial v_E(t,S)}{\partial S}
         +\frac{1}{2}\sigma(t,S)^2S^2\frac{\partial^2 v_E(t,S)}{\partial S^2}=0,
    \end{equation}
    with the terminal condition,
    \begin{equation}
      v_E(T,S) = P(T)\left(\omega[S-K]\right)^+.
    \end{equation}



  \section{Early Exercise Premium Representation}

    Now, let us consider the valuation of American options. Due to the early exercise provision, it is preferable
    to exercise the option and receive the intrinsic value of the option, when the underlying prices moves above or
    below the optimal exercise boundary for a call or put option. Otherwise, the American option is identical to
    its European counterpart. Denote the optimal exercise boundary as $B_T(t)$ for an American option with expiry
    $T$, the discounted American option price can be written as
    \begin{equation}
      v_A(t,S)=P(t)V_A(t,S;T,K,\omega)
              =\begin{cases}
                 \omega P(t)\left(S-K\right), & \mbox{if } \omega S \geq \omega B_T(t),\\
                 v_E(t,S), & \mbox{otherwise}.
               \end{cases}
    \end{equation}
    The dynamics of the discounted American option price can be obtained by the application of Ito's lemma in the
    continuation and the exercise regions,
    \begin{eqnarray}
      &&dv_A(t,S(t)) \nonumber\\
      =&&\mathbf{1}_{\{\omega S(t) \geq \omega B_T(t)\}}\omega\left[P^{\prime}(t)\left(S(t)-K\right)dt
                  +P(t)S(t)\left(-\frac{F^{\prime}(t)}{F(t)}dt+\sigma(t,S(t))dW^{\mathbb Q}(t)\right)\right]\nonumber\\
      +  && \mathbf{1}_{\{\omega S(t)<\omega B_T(t)\}}\bigg[\left(\frac{\partial v_E(t,S(t))}{\partial t}-\frac{F^{\prime}(t)}{F(t)}S(t)\frac{\partial v_E(t,S(t))}{\partial S}
               +\frac{1}{2}\sigma(t,S(t))^2S(t)^2\frac{\partial^2 v_E(t,S(t))}{\partial S^2}\right)dt\nonumber\\
               &&\quad\quad\quad\quad\quad\quad\quad\quad +\sigma(t, S(t))S(t)\frac{\partial v_E(t,S)}{\partial S}dW^{\mathbb Q}(t)\bigg].
    \end{eqnarray}
    Notice the Black-Scholes equation for the discounted European option price, we have
    \begin{eqnarray}
      dv_A(t,S(t))&=&\mathbf{1}_{\{\omega S(t)\geq\omega B_T(t)\}}\omega P(t)
                     \left[\left(\frac{P^{\prime}(t)}{P(t)}-\frac{F^{\prime}(t)}{F(t)}\right)S(t)
                           -\frac{P^{\prime}(t)}{P(t)}K\right]dt\nonumber\\
               && + O\left(dW^{\mathbb Q}(t)\right).
    \end{eqnarray}
    Integrate both sides, and take the risk neutral expectation, the integration with respect to the Brownian motion
    will drop out, we are left with
    \begin{eqnarray}
      &&{\mathbb E}_t^{\mathbb Q}\left[v_A(T,S(T))\right]-v_A(t,S(t)) \nonumber\\
      = &&\omega\int_t^T P(u)\Bigg[\left(\frac{P^{\prime}(u)}{P(u)}-\frac{F^{\prime}(u)}{F(u)}\right)
                          {\mathbb E}_t^{\mathbb Q}\left[S(u)\mathbf{1}_{\{\omega S(u)\geq\omega B_T(u)\}}\right]\nonumber\\
        &&\quad\quad\quad\quad\quad\quad\quad- \frac{P^{\prime}(u)}{P(u)} K
                          {\mathbb E}_t^{\mathbb Q}\left[\mathbf{1}_{\{\omega S(u)\geq\omega B_T(u)\}}\right] \Bigg]du.
    \end{eqnarray}
    At expiry, the payoff of the American option coincides with the European option,
    \begin{equation}
      v_A(T,S(T)) = v_E(T,S(T))=P(T)\left[\omega\left(S(T)-K\right)\right]^+,
    \end{equation}
    the risk neutral expectation of which will actually reproduce the discounted European option price, Eq. (\ref{Euro0}).
    Then, the discounted American option price becomes
    \begin{eqnarray}
      \label{EEP0}
      v_A(t,S(t)) &=& v_E(t,S(t))
      +\omega\int_t^T P(u)\Bigg[\frac{P^{\prime}(u)}{P(u)} K
      {\mathbb E}_t^{\mathbb Q}\left[\mathbf{1}_{\{\omega S(u)\geq\omega B_T(u)\}}\right]\nonumber\\
      &&\quad\quad\quad\quad\quad\quad\quad\quad\quad\ -\left(\frac{P^{\prime}(u)}{P(u)}-\frac{F^{\prime}(u)}{F(u)}\right)
      {\mathbb E}_t^{\mathbb Q}\left[S(u)\mathbf{1}_{\{\omega S(u)\geq\omega B_T(u)\}}\right]\Bigg]du.
    \end{eqnarray}
    This is a generic result, and can be applied to any local volatility model.

  \section{Application to the Black-Scholes Model}

    In the following, we will consider the Black-Scholes model with time-dependent but deterministic volatility
    function, {\it i.e.}, $\sigma(t,S(t))\equiv \sigma(t)$. The discounted European option price is given by
    \begin{eqnarray}
      \label{Euro1}
      v_E(t,S)&=&\omega P(T)\bigg(\frac{S}{F(t,T)}\Phi\left[\omega d_+\left(\frac{S}{F(t,T)K},\Sigma(t,T)\right)\right]\nonumber\\
                         &&\ \ \ \ \ \ \ \ \ \ \ \ \ \ \ \ -K\Phi\left[\omega d_-\left(\frac{S}{F(t,T)K},\Sigma(t,T)\right)\right]\bigg),
    \end{eqnarray}
    where $\omega=\pm 1$ for call/put options, $\Phi(x)$ is the cumulative distribution function of the standard
    normal distribution, and
    \begin{equation}
      d_{\pm}(x, v)=\frac{\log x}{v}\pm\frac{1}{2}v,
    \end{equation}
    with
    \begin{equation}
      \Sigma(t,T)^2=\int_t^T\sigma(u)^2du
    \end{equation}
    Also notice that, for the Black-Scholes model,
    \begin{equation}
      {\mathbb E}_t^{\mathbb Q}\left[S(u)\mathbf{1}_{\{\omega S(u)\geq\omega B_T(u)\}}\right]
       = \frac{S(t)}{F(t,u)}\Phi\left[\omega d_+\left(\frac{S(t)}{F(t,u)B_T(u)}, \Sigma(t,u)\right)\right],
    \end{equation}
    and
    \begin{equation}
      {\mathbb E}_t^{\mathbb Q}\left[\mathbf{1}_{\{\omega S(u)\geq\omega B_T(u)\}}\right]
       = \Phi\left[\omega d_-\left(\frac{S(t)}{F(t,u)B_T(u)}, \Sigma(t,u)\right)\right],
    \end{equation}
    then the discounted American option price can be represented as the sum of the discounted European
    option price and the early exercise premium,
    \begin{eqnarray}
      \label{EEP}
      &&v_A(t,S(t))=v_E(t,S(t))+\omega\int_t^TP(u)\Bigg[\frac{P^{\prime}(u)}{P(u)} K
      \Phi\left[\omega d_-\left(\frac{S(t)}{F(t,u)B_T(u)}, \Sigma(t,u)\right)\right]\nonumber\\
      &&\quad\quad\quad\quad\quad\quad\quad -\left(\frac{P^{\prime}(u)}{P(u)}-\frac{F^{\prime}(u)}{F(u)}\right)
      \frac{S(t)}{F(t,u)}\Phi\left[\omega d_+\left(\frac{S(t)}{F(t,u)B_T(u)}, \Sigma(t,u)\right)\right]\Bigg]du.
    \end{eqnarray}

    The above representation of the American option price requires the knowledge of the optimal
    exercise boundary, which is unknown and must be solved in tandem with the computation of the American
    option price. From Eq. (\ref{EEP}), at the boundary, the American option price becomes its intrinsic
    value,
    \begin{eqnarray}
       \label{Boundary}
       && P(t)\left(B_T(t)-K\right)\nonumber\\
      &=& P(T)\left(\frac{B_T(t)}{F(t,T)}\Phi\left[\omega d_+\left(\frac{B_T(t)}{F(t,T)K},\Sigma(t,T)\right)\right]
                                                      -K\Phi\left[\omega d_-\left(\frac{B_T(t)}{F(t,T)K},\Sigma(t,T)\right)\right]\right)\nonumber\\
       &+&\int_t^TP(u)\Bigg[\frac{P^{\prime}(u)}{P(u)} K
       \Phi\left[\omega d_-\left(\frac{B_T(t)}{F(t,u)B_T(u)}, \Sigma(t,u)\right)\right] \nonumber\\
       &&\quad\quad\quad\quad\quad\quad - \left(\frac{P^{\prime}(u)}{P(u)}-\frac{F^{\prime}(u)}{F(u)}\right)
       \frac{B_T(t)}{F(t,u)}\Phi\left[\omega d_+\left(\frac{B_T(t)}{F(t,u)B_T(u)}, \Sigma(t,u)\right)\right] \Bigg]du.
    \end{eqnarray}
    Using the relations
    \begin{equation}
     \int_t^T \frac{P(u)}{F(u)}\left(\frac{P^{\prime}(u)}{P(u)}-\frac{F^{\prime}(u)}{F(u)}\right)
     = \int_t^T\left(\frac{P(u)}{F(u)}\right)^{\prime}du=\frac{P(T)}{F(T)} - \frac{P(t)}{F(t)},
    \end{equation}
    and
    \begin{equation}
      \int_t^TP^{\prime}(u)du=P(T)-P(t),
    \end{equation}
    the optimal exercise boundary can be determined from the following equation,
    \begin{equation}
      B_T(t)=KF(t,T)\frac{N\left(B_T, \Sigma\right)}{D\left(B_T, \Sigma\right)},
    \end{equation}
    where the numerator $N$ and the denominator $D$ are functionals of the optimal exercise boundary,
    \begin{eqnarray}
      N\left(B_T, \Sigma\right)&=&\Phi\left[-\omega d_-\left(\frac{B_T(t)}{F(t,T)K},\Sigma(t,T)\right)\right]\nonumber\\
      &&-\int_t^T\frac{P^{\prime}(u)}{P(T)}\Phi\left[-\omega d_-\left(\frac{B_T(t)}{F(t,T)B_T(u)},\Sigma(t,u)\right)\right]du,
    \end{eqnarray}
    and
    \begin{eqnarray}
      D\left(B_T, \Sigma\right)&=&\Phi\left[-\omega d_+\left(\frac{B_T(t)}{F(t,T)K},\Sigma(t,T)\right)\right]\nonumber\\
      &&-\int_t^T\frac{F(T)}{P(T)}\left(\frac{P(u)}{F(u)}\right)^{\prime}\Phi\left[-\omega d_+\left(\frac{B_T(t)}{F(t,T)B_T(u)},\Sigma(t,u)\right)\right]du.
    \end{eqnarray}

    From the above formulation, the optimal exercise boundary can be determined by iteration. At the start of
    each iteration, with the knowledge of the optimal exercise boundary $B_T^{(j)}(t)$ from the previous iteration,
    a new boundary $B_T^{(j+1)}(t)$ can be found. This process can be repeated until convergence is reached.





\begin{thebibliography}{99}
  \bibitem{ALO}
    Leif Andersen, Mark Lake and Dimitri Offengenden,
    \href{https://www.risk.net/journal-of-computational-finance/2464632/high-performance-american-option-pricing}
    {\it High-performance American option pricing},
    Journal of Computational Finance 20(1):39-87 (2016).
\end{thebibliography}


\end{document}
